\documentclass[class=book, crop=false, oneside]{standalone}
\usepackage[subpreambles=true]{standalone}

\usepackage{../../style}

% arara: pdflatex: { synctex: yes, shell: yes }
% arara: latexmk: { clean: partial }
%! arara: clean: { extensions: [sta] }
\begin{document}
\chapter{Ancora sui programmi Assembly}
Si consideri quest'appendice come un'estensione del capitolo~\ref{ch:asm}; qui illustreremo tutti gli esempi presenti nella dispensa del professor Abeni, dimodoché anche il lettore più curioso possa trovare pane per i suoi denti. Enjoy.

\section{Ordinamento di array}
Mostriamo ora le diverse traduzioni della seguente funzione \emph{insert sort} scritta in C:

\begin{minted}[linenos, breaklines, tabsize = 4]{c}
void sposta(int v[], int i){
	int j;
	int appoggio;

	appoggio = v[i];
	j = i - 1;

	while((j >= 0) && (v[j] > appoggio)){
		v[j +	1] = v[j];
		j = j - 1;
	}

	v[j + 1]
}

void ordina(int v[], int n){
	int i = 1;

	while (i < n){
		sposta(v, i);
		i = i + 1;
	}
}
\end{minted}

\subsubsection{Implementazione MIPS}
Prima versione, "fatta a mano":
\begin{minted}[linenos, breaklines, tabsize = 4]{asm}
sposta:
		# a0 = v, a1 = i
		# t0 = j * 4, t1 = appoggio
	sll $t0, $a1, 2
	add $t2, $a0, $t0  # t2 = &v[i]
	lw $t1, 0($t2)     # appoggio = v[i]
	addi $t0, $t0, -4

ciclo:
	slt $t3, $t0, $zero  # t3 = 1 se j < 0
	bne $t3, $zero, out  # salta a out se j < 0
	add $a2, $a0, $t0    # t2 = &v[j]
	lw $t4, 0($t2)       # t4 = v[j]
	slt $t3, $t1 , $t4   # t3=1 se t1<t4 (appoggio < v[j])
	beq $t3, $zero, out  # esci se appoggio >= v[j]
	addi $t5, $t0, 4     # t5 = (j + 1) * 4
	add $t2, $a0, $t5    # t2 = &v[j + 1]
	sw $t4, 0($t2)
	addi $t0, $t0, -4
	j ciclo

out:
	addi $t5, $t0, 4
	add $t2, $a0, $t5    # t2 = &v[j + 1]
	sw $t1, 0($t2)
	jr $ra

ordina:
	# a0 = v, a1 = n
	# s0 = i
	addi $sp, $sp, -12
	sw $s0, 0($sp)
	sw $ra, 4($sp)
	sw $a1, 8($sp)

	addi $s0, $zero, 1  # i = 1

loop_ordina:
	slt $t0, $s0, $a1   # t0 = 1 se i < n
	beq $t0, $zero, out_ordina
	add $a1, $s0, $zero
	jal sposta
	lw $a1, 8($sp)
	addi $s0, $s0, 1
	j loop_ordina

out_ordina:
	lw $s0, 0($sp)
	lw $ra, 4($sp)
	addi $sp, $sp, 12
	jr $ra
\end{minted}

Ecco invece l'implementazione, più complessa, che ci presenta il nostro fido \mintinline{batch}{gcc}:

\begin{minted}[linenos, breaklines, tabsize = 4]{asm}
sposta:
	sll $v0, $a1, 2
	addu $v0, $a0, $v0
	lw $t0, 0($v0)
	addiu $v0, $a1, -1
	bltz $v0, L2
	move $a2, $v0
	sll $v1, $v0, 2
	addu $v1, $a0, $v1
	lw $v1, 0($v1)
	slt $a3, $t0, $v1
	beq $a3, $zero, L2
	sll $a1, $a1, 2
	addu $a1, $a0, $a1
	li $t1 , -1   # 0xffffffffffffffff

L3:
	addiu $a2, $a2, 1
	sll $a2, $a2, 2
	addu $a2, $a0, $a2
	sw $v1, 0($a2)
	addiu $v0, $v0, -1
	beq $v0, $t1, L2
	move $a2, $v0
	addiu $a1, $a1, -4
	lw $v1, 4($a1)
	slt $a3, $t0, $v1
	bne $a3, $zero, L3

L2:
	addiu $v0, $v0, 1
	sll $v0, $v0, 2
	addu $a0, $a0, $v0
	sw $t0, 0($a0)
	jr $ra

ordina:
	addiu $sp, $sp, -16
	sw $ra, 12($sp)
	sw $s2, 8($sp)
	sw $s1, 4($sp)
	sw $s0, 0($sp)
	move $s1, $a1
	slt $v0, $a1, 2
	bne $v0, $zero, L5
	move $s2, $a0
	li $s0, 1	 # 0x1

L7:
	move $a0, $s2
	move $a1, $s0
	jal sposta
	addiu $s0, $s0, 1
	bne $s0, $s1, L7

L5:
	lw $ra, 12($sp)
	lw $s2, 8($sp)
	lw $s1, 4($sp)
	lw $s0, 0($sp)
	addiu $sp, $sp, 16
	jr $ra
\end{minted}

\subsubsection{Implementazione x86}
Ecco una possibile implementazione Intel, sempre artigianale:
\begin{minted}[linenos, breaklines, tabsize = 4]{gas}
sposta:
		# rdi = v, rsi = i
		# rax = j , r10d = appoggio
	movq %rsi, %rax
	movl (%rdi, %rax, 4), %r10d	 # appoggio = v[i]
	dec %rax

ciclo:
	cmpq $0, %rax                # confronta 0 e rax
	jl out                       # esci se j < 0
	movl (%rdi, %rax, 4), %r11d  # metti v[j] in %r11
	cmpl %r10d , %r11d           # confronta v[j] e appoggio
	jle out                      # se v[j] < appoggio, esci
	movl %r11d, 4(%rdi, %rax  4)
	dec %rax
	jmp ciclo

out:
	movl %r10d, 4(%rdi, %rax, 4)
	ret

ordina:
	# rdi = v, rsi = n
	# rbx = i
	pushq %rbx
	movq $1, %rbx

loop_ordina:
	cmp %rbx , %rsi
	jle out_ordina
	pushq %rsi
	movq %rbx, %rsi
	call sposta
	popq %rsi
	inc %rbx
	jmp loop_ordina

out_ordina:
	popq %rbx
	ret
\end{minted}

Implementazione di \mintinline{batch}{gcc}:

\begin{minted}[linenos, breaklines, tabsize = 4]{gas}
sposta:
	movslq %esi, %rax
	leaq 0(%rax, 4 ), %r8
	movl (%rdi, %rax, 4), %ecx
	subl $1, %esi
	js L2
	movslq %esi, %rdx
	movl -4(%rdi, %r8), %eax
	cmpl %eax, %ecx
	jge L2

L4:
	movl %eax, 4(%rdi, %rdx, 4)
	subl $1, %esi
	cmpl $-1, %esi
	je L2
	movslq %esi, %rdx
	movl (%rdi, %rdx, 4), %eax
	cmpl %eax, %ecx
	jl L4

L2:
	movslq %esi, %rsi
	movl %ecx, 4(%rdi, %rsi, 4)
	ret

ordina:
	cmpl $1, %esi
	jle L12
	pushq %r12
	pushq %rbp
	pushq %rbx
	movl %esi, %ebp
	movq %rdi, %r12
	movl $1, %ebx

L8:
	movl %ebx, %esi
	movq %r12, %rdi
	call sposta
	addl $1, %ebx
	cmpl %ebx, %ebp
	jne L8
	popq %rbx
	popq %rbp
	popq %r12

L12:
	ret
\end{minted}

\subsubsection{Implementazione ARM}
\begin{minted}[linenos, breaklines, tabsize = 4]{arm}
sposta:
	mov r2, r1, asl #2
	add r3 , r0, r2
	ldr ip, [r0, r1, asl #2]
	subs r1, r1, #1
	bmi L2
	ldr r2, [r3, #-4]
	cmp ip, r2
	bge L2

L3:
	str r2, [r3], #-4
	sub r1, r1, #1
	cmn r1, #1
	beq L2
	ldr r2, [r3, #-4]
	cmp ip, r2
	blt L3

L2:
	add r1, r1, #1
	str ip, [r0, r1, asl #2]
	bx lr

ordina:
	cmp r1, #1
	bxle lr
	stmfd sp!, {r4, r5, r6, lr}
	mov r5, r1
	mov r6, r0
	mov r4, #1

L8:
	mov r1, r4
	mov r0, r6
	bl sposta
	add r4, r4, #1
	cmp r5, r4
	bne L8
	ldmfd sp!, {r4, r5, r6, pc}
\end{minted}
\section{Funzione \(2^{n}\)}

\subsection*{Via iterativa}

\begin{minted}[linenos, breaklines, tabsize = 4]{c}
int potenza_due(int n){
	if (n == 0){
		return 1;
	} else {
		int ret = 1;
		int i;
		for (i=0; i<n; i++){
			ret = ret * 2;
		}

		return ret;
		}
}
\end{minted}

\subsubsection{Implementazione MIPS}

\begin{minted}[linenos, breaklines, tabsize = 4]{asm}
potenza_due:
	beq $a0, $zero, L4
	blez $a0, L5
	move $v1, $zero
	li $v0, 1  # 0x1

L3:
	sll $v0, $v0, 1
	addiu $v1, $v1, 1
	bne $v1, $a0, L3
	jr $ra

L4:
	li $v0, 1  # 0x1
	jr $ra

L5:
	li $v0, 1  # 0x1
	jr $ra
\end{minted}

\subsubsection{Implementazione x86}

\begin{minted}[linenos, breaklines, tabsize = 4]{gas}
potenza_due:
	testl %edi, %edi
	jle L4
	movl $0, %edx
	movl $1, %eax

L3:
	addl %eax, %eax
	addl $1, %edx
	cmpl %edx, %edi
	jne L3
	ret

L4:
	movl $1, %eax
	ret
\end{minted}

\subsubsection{Implementazione ARM}

\begin{minted}[linenos, breaklines, tabsize = 4]{arm}
potenza_due:
	subs r2, r0, #0
	ble L4
	mov r3, #0
	mov r0, #1
L3:
	mov r0, r0, asl #1
	add r3, r3, #1
	cmp r2, r3
	bne L3
	bx lr
L4:
	mov r0, #1
	bx lr
\end{minted}

\subsection*{Via ricorsiva}

\begin{minted}[linenos, breaklines, tabsize = 4]{c}
int potenza_due (int n){
	if (n < 1){
	return 1;
	} else {
		return 2 * potenza_due(n - 1);
	}
}
\end{minted}

\subsubsection{Implementazione MIPS}

\begin{minted}[linenos, breaklines, tabsize = 4]{asm}
potenza_due:
	blez $a0, L3
	addiu $sp, $sp, -8
	sw $ra, 4($sp)
	addiu $a0, $a0, -1
	jal potenza_due
	sll $v0, $v0, 1
	j L2

L3:
	li $v0, 1  # 0x1
	jr $ra

L2:
	lw $ra, 4($sp)
	addiu $sp, $sp, 8
	jr $ra
\end{minted}

\subsubsection{Implementazione x86}

\begin{minted}[linenos, breaklines, tabsize = 4]{gas}
potenza_due:
	movl $1, %eax
	testl %edi, %edi
	jle L6
	subq $8, %rsp
	subl $1, %edi
	call potenza_due
	addl %eax, %eax
	addq $8, %rsp

L6:
	ret
\end{minted}

\subsubsection{Implementazione ARM}

\begin{minted}[linenos, breaklines, tabsize = 4]{arm}
potenza_due:
	cmp r0, #0
	ble L3
	stmfd sp!, {r4, lr}
	sub r0, r0, #1
	bl potenza_due
	mov r0, r0, asl #1
	ldmfd sp!, {r4, pc}

L3:
	mov r0, #1
	bx lr
\end{minted}

\section{Successione di Fibonacci}
Presentiamo ora l'implementazione di una funzione che calcola l'\(n\)-esimo numero di Fibonacci:
\begin{minted}[linenos, breaklines, tabsize = 4]{c}
int fibonacci(int n){
	if (n < 2){
		return 1;
	} else {
		return 3 * fibonacci(n - 1) - fibonacci(n - 2);
	}
}
\end{minted}

\subsubsection{Implementazione MIPS}

\begin{minted}[linenos, breaklines, tabsize = 4]{asm}
fibonacci:
	addiu $sp, $sp, -16
	sw $ra, 12($sp)
	sw $s1, 8($sp)
	sw $s0, 4($sp)
	move $s0, $a0
	slt $v0, $a0, 2
	bne $v0, $zero, L3
	addiu $a0, $a0, -1
	jal fibonacci
	move $s1, $v0
	addiu $a0, $s0, -2
	jal fibonacci
	sll $v1, $s1, 1
	addu $s1, $v1, $s1
	subu $v0, $s1, $v0
	j L2
L3:
	li $v0, 1	 # 0x1
L2 :
	lw $ra, 12($sp)
	lw $s1, 8($sp)
	lw $s0, 4($sp)
	addiu $sp, $sp, 16
	jr $ra
\end{minted}

\subsubsection{Implementazione x86}

\begin{minted}[linenos, breaklines, tabsize = 4]{gas}
fibonacci:
	movl $1, %eax
	cmpl $1, %edi
	jle L6
	pushq %rbp
	pushq %rbx
	subq $8, %rsp
	movl %edi, %ebx
	leal -1(%rdi), %edi
	call fibonacci
	movl %eax, %ebp
	leal -2(%rbx), %edi
	call fibonacci
	leal 0(%rbp, %rbp, 2), %edx
	subl %eax, %edx
	movl %edx, %eax
	addq $8, %rsp
	popq %rbx
	popq %rbp
L6:
	ret
\end{minted}

\subsubsection{Implementazione ARM}

\begin{minted}[linenos, breaklines, tabsize = 4]{arm}
fibonacci:
	cmp r0, #1
	ble L3
	stmfd sp!, {r4, r5, r6, lr}
	mov r5, r0
	sub r0, r0, #1
	bl fibonacci
	mov r4, r0
	sub r0, r5, #2
	bl fibonacci
	add r4, r4, r4, lsl #1
	rsb r0, r0, r4
	ldmfd sp!, {r4, r5, r6, pc}
L3:
	mov r0, #1
	bx lr
\end{minted}

\section{Serie tail recursive}
Vediamo ora una funzione che calcola i primi \(n\) numeri naturali, in versione tail recursive:
\begin{minted}[linenos, breaklines, tabsize = 4]{c}
int sum(int n, int acc){
	if (n > 0){
		return sum(n - 1, acc + n);
	} else {
		return acc;
	}
}
\end{minted}

\subsection*{Senza ottimizzazione di chiamate in coda}
Vediamone inizialmente una traduzione fatta senza l'eliminazione delle chiamate ricorsive:
\subsubsection{Implementazione MIPS}

\begin{minted}[linenos, breaklines, tabsize = 4]{asm}
sum:
	move $v1, $a0
	move $v0, $a1
	blez $a0, L5
	addiu $sp, $sp, -8
	sw $ra, 4($sp)
	addiu $a0, $a0, -1
	addu $a1, $a1 , $v1
	jal sum
	lw $ra, 4($sp)
	addiu $sp, $sp,8
L5:
	jr $ra
\end{minted}

\subsubsection{Implementazione x86}

\begin{minted}[linenos, breaklines, tabsize = 4]{gas}
sum :
	movl %esi, %eax
	testl %edi, %edi
	jle L6
	subq $8, %rsp
	addl %edi, %esi
	subl $1, %edi
	call sum
	addq $8, %rsp
L6:
	ret
\end{minted}

\subsubsection{Implementazione ARM}

\begin{minted}[linenos, breaklines, tabsize = 4]{arm}
sum:
	cmp r0, #0
	ble L3
	stmfd sp!, {r4, lr}
	add r1, r0, r1
	sub r0, r0, #1
	bl sum
	ldmfd sp!, {r4, pc}
L3:
	mov r0, r1
	bx lr
\end{minted}

\subsection*{Chiamate in coda ottimizzate}
Proponiamo ora la traduzione che \mintinline{batch}{gcc} esegue attivando il comando \mintinline{batch}{-foptimize-sibling-calls}, la cui funzione è ottimizzare le chiamate in coda:
\subsubsection{Implementazione MIPS}

\begin{minted}[linenos, breaklines, tabsize = 4]{asm}
sum:
	move $v0, $a1
	blez $a0, L2
L4:
	addu $v0, $v0, $a0
	addiu $a0, $a0, -1
	bne $a0, $zero, L4
L2:
	jr $ra
\end{minted}

\subsubsection{Implementazione x86}

\begin{minted}[linenos, breaklines, tabsize = 4]{gas}
sum:
	movl %esi, %eax
	testl %edi , %edi
	jle L2
L4:
	addl %edi, %eax
	subl $1, %edi
	jne L4
L2:
	ret
\end{minted}

\subsubsection{Implementazione ARM}

\begin{minted}[linenos, breaklines, tabsize = 4]{arm}
sum:
	cmp r0, #0
	ble L2
L4:
	sub r3, r0, #1
	add r1, r1, r0
	mov r0, r3
	cmp r3, #0
	bne L4
L2:
	mov r0, r1
	bx lr
\end{minted}

\end{document}
