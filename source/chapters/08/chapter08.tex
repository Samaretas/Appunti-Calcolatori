\documentclass[class=book, crop=false, oneside]{standalone}
\usepackage[subpreambles=true]{standalone}

\usepackage{../../style}
\usepackage{../../set-citations}

\graphicspath{{./assets/images/}}
\newmintinline{arm}{}

% arara: pdflatex: { synctex: yes, shell: yes }
% arara: latexmk: { clean: partial }
\begin{document}

\chapter{L'architettura ARM}\begin{fquote}[Alan J. Perlis]When someone says: "I want a programming language in which I need only say what I wish done", give him a lollipop.
 \end{fquote}

\section{Introduzione}
L'architettura \emph{ARM} nasce negli anni Ottanta come architettura \emph{RISC} pragmatica, che tenta di migliorare la suddetta architettura implementando delle istruzioni e delle strategie appartenenti ad \emph{Intel}: è proprio questo il suo punto di forza.

Un'altra caratteristica peculiare dell'\emph{ARM} è la presenza di una modalità \emph{low power}, che limita di molto il consumo di energia restringendo le istruzioni utilizzabili solo a quelle più semplici (e quindi meno costose in termini di risorse). È proprio per questa ragione che l'\emph{ARM} ha avuto un'enorme diffusione come architettura per i dispositivi \emph{mobile} e \emph{wearable} oltre che per i sistemi \emph{embedded}.

Per amor di precisione ci teniamo a specificare che in realtà "architettura ARM" non è una dicitura univoca in quanto esistono più \emph{CPU} \emph{ARM} con \emph{ISA} differenti (per non parlare del numero spropositato di \emph{ABI} anche molto diverse tra loro che si possono trovare per questa architettura).

\section{Registri e relativo utilizzo}
In ARM ci sono 16 registri a 32 bit e sono nominati da \register{r0} a \register{r15} (da notare che in ARM per indicare un registro non serve un prefisso). Alcuni di essi hanno generalmente funzioni specifiche, ovvero:

\begin{itemize}
	\item \register{r13}: \emph{stack pointer}, contiene il puntatore allo stack;
	\item \register{r14}: \emph{link register}, che contiene l'indirizzo di ritorno delle subroutine (come \register{\$ra} in MIPS);
	\item \register{r15}: \emph{program counter} (nei bit più significativi contiene anche dei flags).
\end{itemize}
Nonostante sia molto diffuso, questo standard non è assoluto: di fatto esistono \emph{ABI} che implementano l'utilizzo dei registri in modo diverso.

\section{Le convenzioni di chiamata}
Le convenzioni di chiamata, che vengono specificate appositamente dall'\emph{ABI}, servono per "mettere d'accordo" diversi compilatori/librerie e altre parti del sistema operativo: sono esse che determinano se passare i parametri solo tramite registri o anche via stack, quali registri preservare, quali registri utilizzare liberamente e come gestire l'indirizzo di ritorno. A seguire ecco cosa stabilisce l'ABI più diffusa del mondo ARM:
\begin{itemize}
	\item \register{r0}, \register{r1}, \register{r2}, \register{r3}, \register{r12} sono registri temporanei;
	\item da \register{r4} a \register{r11} sono da preservare, con l'eccezione di \register{r9} che in alcune ABI non viene preservato.
\end{itemize}

\section{Le istruzioni ARM}
Le istruzioni dell'Assembly ARM sono molto simili a quelle del buon vecchio MIPS, tuttavia sono meno regolari in quanto alcune di esse presentano solamente due argomenti; è inoltre notevole il fatto che, nonostante venga classificato come architettura RISC, ARM presenti una lista di istruzioni piuttosto corposa.

Come nel MIPS, le istruzioni a tre argomenti prevedono il primo come destinazione e i successivi come operandi; inoltre l'operando di sinistra dev'essere un registro mentre quello di destra può essere sia un registro che un valore immediato. Da notare che i dati in memoria non possono essere usati direttamente come operandi: risolveremo questo problema in seguito, introducendo delle operazioni specifiche \emph{load/store register}.

Tutte le istruzioni in \emph{ARM} permettono esecuzione condizionale grazie all'inserimento di determinati suffissi al termine del nome identificativo dell'istruzione: l'esecuzione dell'istruzione stessa avviene solo se i bit del registro di flag corrispondono alla condizione desiderata; qui è riportata una tabella con i principali suffissi ed il loro funzionamento:

\vspace{0.4cm}
\begin{table}[H]
	\centering
	\subimport{assets/tables/}{condizionali.tex}
	\caption{Suffissi condizionali ARM}
\end{table}
I flag stessi possono essere modificati da apposite operazioni come \opcode{cmp} (\emph{compare}), o da operazioni a cui si aggiunge il suffisso opzionale \suffix{-s}, che va a settare i bit del registro di flag se si registrano determinate condizioni (ad esempio un overflow durante una somma).

Lo schema base di un'istruzione è il seguente:
\begin{center}
	\texttt{<opcode>[<-cond>][<-s>] <rd>, <rl>, <r*>}
\end{center}
dove \suffix{<-cond>} ed \suffix{<-s>} sono facoltativi, \register{<rd>} è il registro di destinazione, \register{<rl>} è il registro di sinistra e \register{<r*>} è il registro di destra. Si noti inoltre che \register{<r*>} può essere sia un registro che un valore immediato che un registro su cui si effettuano operazioni di modifica (si veda ~\ref{sec:modifiche}). Tuttavia, a differenza di MIPS, ARM non ha bisogno di due opcode distinti per definire l'operazione di somma: in ARM è possibile utilizzare \opcode{add} sia per sommare due registri che per sommare un valore immediato ad un registro.

Se il secondo operando è un registro, si possono inserire al termine dell'operazione dei comandi ulteriori che permettono di eseguire operazioni di shifting o rotazione sul secondo operando (che avvengono prima dell'operazione), ad esempio \opcode{lsl}/\opcode{asl}, \opcode{lsr}, \opcode{asr}, \opcode{ror}, \opcode{rrx} (si veda sempre ~\ref{sec:modifiche}).

\subsection*{Le istruzioni più comuni}
Vediamo ora una lista delle operazioni più comuni:

\begin{itemize}
	\item \opcode{add}/\opcode{adc}: rispettivamente somma e somma con resto;
	\item \opcode{sub}/\opcode{sbc}: rispettivamente sottrazione e sottrazione con riporto;
	\item \opcode{rsb}/\opcode{rsc}: \emph{reverse sub/reverse sub with carry} che permettono una sottrazione invertendo il primo ed il secondo operando. In questo modo risulta possibile sottrarre un registro ad un numero;
	\item \opcode{and}/\opcode{orr}/\opcode{eor}/\opcode{bic}: operazioni booleane bit a bit. Si noti che \mintinline{arm}{bic r0, r1, r} calcola \(\register{r0} = (\register{r1} \textrm{ and } (\textrm{not } \register{r}))\);
	\item \opcode{mul}/\opcode{mla} e simili sono per varie forme di moltiplicazione\footnote{Non esistono comandi per la divisione in assembly ARM.};
	\item \opcode{b}/\opcode{bl} rispettivamente \emph{branch} e \emph{branch and link} (da notare che è possibile aggiungere qualsiasi suffisso condizionale);
	\item \opcode{bx} è implementato solo in alcune CPU, corrisponde a \mintinline{arm}{move r15, r};
	\item \opcode{cmp} setta i flags comparando due operandi, ma senza risultato; da usarsi prima di salti condizionali. Funzionano similmente anche \opcode{tst} (attraverso un and) e \opcode{teq} (attraverso uno xor);
	\item \opcode{ldr}/\opcode{str} sono rispettivamente \emph{load register} e \emph{store register}. Rispecchiano rispettivamente \texttt{lw} e \texttt{sw} dell'assembly MIPS;
	\item \opcode{mov}/\opcode{mvn}:  equivalente agli omonimi move e move not in MIPS; move not sposta il complemento a 1 del registro sorgente (utilizzata per implementare il not).
\end{itemize}

\subsection{Possibilli operazioni sull'operando <r*>}\label{sec:modifiche}
Prima di mostrare le operazioni possibili su \texttt{<r*>}, ci teniamo a specificare che queste sono operazioni di shifting e di rotating e vengono applicate al secondo operando prima dello svolgimento dell'operazione descritta dall'istruzione.

Ecco un esempio di un'operazione in diverse versioni, per capire al meglio la potenza di ARM:
\begin{minted}[linenos, tabsize=4]{arm}
add r0, r1, #1
adds r0, r1, r2
addeq r0, r1, r2, lsl #2
addeqs r0, r1, r2, lsl r3
\end{minted}
Analizziamo le righe di codice una ad una:
\begin{enumerate}
	\item \register{r0} \(=\) \registerOp{r1}{+}{1};
	\item \register{r0} \(=\) \registerOp{r1}{+}{r2} e setta dei flag se necessario;
	\item \register{r0} \(=\) \registerOp{r1}{+}{r2}: esegue la somma solo se il flag \mintinline{arm}{eq} è settato, prima dell'operazione esegue uno shift (temporaneo) a sinistra di 2 del registro \register{r2};
	\item \register{r0} \(=\) \registerOp{r1}{+}{r2}: esegue la somma solo se il flag \mintinline{arm}{eq} è settato, setta flag se necessario, prima dell'operazione esegue uno shift (temporaneo) a sinistra di \register{r3} del registro \register{r2}.
\end{enumerate}

\subsection*{Istruzioni di load/store multiple}
L'Assembly ARM permette il salvataggio in memoria o il caricamento da memoria di più registri con una sola istruzione, \mintinline{arm}{ldm} e \mintinline{arm}{stm} per l'appunto. Le istruzioni appena citate hanno la seguente struttura:
\begin{center}
	\texttt{ldm [-mode][-cond] rb[!], (lista dei registri)}
\end{center}
dove \register{rb} è il registro che contiene la base, \suffix{-cond} è l'eventuale suffisso condizionale di cui si è già discusso in precedenza, mentre il suffisso (facoltativo) \suffix{mode} può essere:
\begin{itemize}
	\item \suffix{ia}: \emph{increment after}, incrementa il registro base dopo l'esecuzione di \opcode{ldm};
	\item \suffix{ib}: \emph{increment before}, incrementa il registro base prima dell'esecuzione;
	\item \suffix{da}: \emph{decrement after}, decrementa il registro base dopo l'esecuzione;
	\item \suffix{db}: \emph{decrement before}, decrementa il registro base prima dell'esecuzione.
\end{itemize}

Il \suffix{!}, facoltativo anch'esso, serve per rendere definitive le modifiche apportate dai suffissi appena elencati, dopo l'operazione.

\section{Modalità di indirizzamento in ARM}
Introduciamo ora le modalità d'indirizzamento in ARM, ricordando che si tratta di istruzioni che rendono possibile modificare l'indirizzo indicato dal secondo registro di operazioni come \mintinline{arm}{ldr} e \mintinline{arm}{str}; si tenga presente che le operazioni di indirizzamento sono diverse dalle operazioni di shifting o rotating.

In modo semplificato esistono due metodologie di indirizzamento in ARM, ovvero \texttt{base}\(+\)\texttt{offset} (spiazzamento, valore immediato) o \texttt{base}\(+\)\texttt{index} (indice, eventualmente scalato). La struttura dell'istruzione è la seguente:

\begin{minted}[tabsize=4, linenos]{arm}
ldr rd, rb (offset)         ; base + spiazzamento
ldr rd, rb (index) [shift]  ; base + indice shiftato
\end{minted}
dove \register{rd} è il registro destinazione, \register{rb} è il registro contenente l'indirizzo base, \register{offset} è un immediato codificato su 12 bit e \register{index} è un valore in registro, eventualmente shiftato o ruotato.

Una possibilità molto interessante offerta da ARM è l'utilizzo dei meccanismi di \emph{pre indexing} e \emph{post indexing}, vediamo come funzionano:

\begin{itemize}
	\item \emph{pre indexing con offset}: \texttt{[rb, \#i]!}, in questo caso la somma tra \register{rb} e \register{i} viene  effettuata prima dell'indirizzamento, il \register{!} è facoltativo e se è presente finalizza la somma (il valore di \register{rb} dopo l'esecuzione è dato da \registerOp{rb}{+}{i});
	\item \emph{post indexing con offset}: \texttt{[rb], \#i!}, in questo caso la somma tra \register{rb} e \register{i} viene effettuata dopo l'indirizzamento, il \suffix{!} se è presente finalizza la somma (il valore di \register{rb} dopo l'esecuzione è dato da \registerOp{rb}{+}{i});
	\item \emph{pre indexing con indice e shift}: \texttt{[rb, ri, shift]!}, in questo caso la somma tra \register{rb} e \register{ri} shiftato (si ricorda che lo shift è facoltativo) viene  effettuata prima dell'indirizzamento, il \suffix{!} se è presente finalizza la somma;
	\item \emph{post indexing con indice e shift}: \texttt{[rb], ri, shift !}, in questo caso la somma tra \register{rb} e \register{ri} eventualmente shiftato viene effettuata dopo dell'indirizzamento.
\end{itemize}
Tirando le somme sui metodi di indirizzamento, si può dire che sono senz'altro più potenti di quelli offerti da \emph{MIPS}, soprattutto grazie alla possibilità di aggiornare automaticamente il registro di base, che diventa super comoda quando si devono gestire degli array. Rispetto a x86, invece, ARM ha la limitazione di non poter usare contemporaneamente spiazzamento ed indice.

\end{document}
