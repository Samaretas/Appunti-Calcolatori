\documentclass[class=book, crop=false, oneside]{standalone}
\usepackage[subpreambles=true]{standalone}

\usepackage{../../style}

\graphicspath{{./assets/images/}}
\newmintinline{asm}{}

% arara: pdflatex: { synctex: yes, shell: yes }
% arara: latexmk: { clean: partial }
\begin{document}

\chapter{Architettura ARM}

\section{Introduzione}
L'architettura \emph{ARM} nasce negli anni Ottanta come architettura \emph{RISC} pragmatica, che tenta di migliorare la suddetta architettura implementando delle istruzioni e delle strategie appartenenti ad \emph{Intel}, è proprio questo il suo punto di forza.

Un'altra caratteristica peculiare dell'\emph{ARM} è la presenza di una modalità \emph{low power}, che limita di molto il consumo di energia restringendo le istruzioni utilizzabili solo a quelle più semplici (e quindi meno costose). È propio per questa ragione che l'\emph{ARM} ha avuto un'enorme diffusione come architettura per i dispositivi \emph{mobile} e \emph{wearable} oltre che per i sistemi \emph{embedded}.

Per amor di precisione ci teniamo a specificare che in realtà "architettura ARM" non è una dicitura univoca in quanto esistono più \emph{CPU} \emph{ARM} con \emph{ISA} differenti, per non parlare del numero spropositato di \emph{ABI} anche molto diverse tra loro che si possono trovare per questa architettura.

\section{Registri e relativo utilizzo}
In ARM i registri sono 16 a 32 bit (anche se ultimamente se ne vedono pure a 64 bit... \emph{non ci sono più i registri di una volta}). I registri hanno sono nominati da \register{r0} a \register{r15} (è scoparso l'orribile simbolo di demarcazione \$) ed alcuni di essi hanno (generalmente) funzioni specifiche, ovvero:

\begin{itemize}
	\item \register{r13}: \emph{stack pointer} (che contiene il puntatore allo stack);
	\item \register{r14}: \emph{link register} (che contiene l'indirizzo di ritorno delle subroutine, come \register{\$ra} in MIPS);
	\item \register{r15}: \emph{program counter} (nei bit più significativi contiene anche dei flags).
\end{itemize}
In realtà questo standard, pur essendo molto diffuso, non è assoluto, di fatto esistono \emph{ABI} che implementano l'utilizzo dei registri in modo diverso.

\section{Le convenzioni di chiamata}
Le convenzioni di chiamata servono appunto per "mettere d'accordo" diversi compilatori/librerie e altre parti del sistema operativo e vengono specificate appositamente dall'\emph{ABI}: sono esse che determinano come passare i parametri (se solo tramite registri o anche via stack), quali registri preservare, quali registri utilizzare liberamente e come gestire l'indirizzo di ritorno. A seguire ecco cosa stabilisce l'\emph{ABI} più diffusa del mondo \emph{ARM}:
\begin{itemize}
	\item \register{r0}, \register{r1}, \register{r2}, \register{r3}, \register{r12} sono registri temporanei;
	\item da \register{r4} a \register{r11} sono da preservare, con l'eccezione di \register{r9} che in alcuni ABI non viene preservato.
\end{itemize}

\section{Le istruzioni ARM}
Le istruzioni dell'Assembly \emph{ARM} sono molto simili a quelle del buon vecchio \emph{MIPS}, tuttavia sono meno regolari in quanto alcune di esse presentano solamente due argomenti; è inoltre notevole il fatto che nonostante la dicitura \emph{RISC} l'\emph{ARM} presenti una lunga lista di istruzioni.

Come nel \emph{MIPS}, le istruzioni a tre argomenti prevedono il primo come destinazione e i seguenti come operandi, l'operando di sinistra dev'essere un registro mentre quello di destra può essere sia un registro che un valore immediato. Da notare che non si possono usare operandi in memoria, per sfruttare questa vedremo in seguito l'utilizzo delle operazioni specifiche \emph{load/store register}.

Tutte le istruzioni in \emph{ARM} permettono esecuzione condizionale grazie all'inserimento di determinati suffissi al termine del identificativo dell'operazione, che comportano l'esecuzione dell'operazione stessa solo se i bit del registro di flag corrispondono alla condizione desiderata; qui è riportata una tabella con i principali suffissi ed il loro funzionamento:

\vspace{0.4cm}
\begin{table}[H]
	\centering
	\subimport{assets/tables/}{condizionali.tex}
	\caption{Suffissi condizionali ARM}
\end{table}
I flag stessi possono essere modificati da apposite operazioni come \opcode{cmp} (\emph{compare}), o da operazioni a cui si aggiunge il suffisso opzionale \suffix{-s}, che va a settare i bit del registro di flag se si registrano determinate condizioni (ad esempio un overflow durante una somma).

Lo schema base di un'operazione è il seguente:
\begin{center}
	\texttt{<opcode>[<-cond>][<-s>] <rd>, <rl>, <r*>}
\end{center}
dove \suffix{<-cond>} ed \suffix{<-s>} sono facoltativi, \register{<rd>} è il registro di destinazione, \register{<rl>} è il registro di sinistra e \register{<r*>} è il registro di destra. Si noti inoltre che \register{<r*>} può essere sia un registro che un valore immediato che un registro su cui si effettuano operazioni di modifica (si veda ~\ref{sec:modifiche}). Tuttavia, a differenza di \emph{MIPS}, \emph{ARM} non ha bisogno di due opcode distinti per definire l'operazione di somma: in \emph{ARM} è possibile utilizzare \opcode{add} sia per sommare due registri, che per sommare un valore immediato ad un registro.

Se il secondo operando è un registro, si possono inserire al termine dell'operazione dei comandi ulteriori che permettono di eseguire operazioni di shift o rotazione sul secondo operando (solo l'avvenimento dell'operazione), ad esempio \opcode{lsl}/\opcode{asl}, \opcode{lsr}, \opcode{asr}, \opcode{ror}, \opcode{rrx} (si veda sempre ~\ref{sec:modifiche}).

\subsection*{Le istruzioni più comuni}
Vediamo ora una lista delle operazioni più comuni:

\begin{itemize}
	\item \opcode{add}/\opcode{adc}: rispettivamente somma e somma con resto;
	\item \opcode{sub}/\opcode{sbc}: rispettivamente sottrazione e sottrazione con riporto;
	\item \opcode{rsb}/\opcode{rsc}: \emph{reverse sub/reverse sub with carry} che permettono una sottrazione invertendo il primo ed il secondo operando. In questo modo risulta possibile sottrarre un registro ad un numero;
	\item \opcode{and}/\opcode{orr}/\opcode{eor}/\opcode{bic}: operazioni booleane bit a bit. Si noti che \mintinline{asm}{bic r0, r1, r} calcola \(\register{r0} = (\register{r1} \textrm{ and } (\textrm{not } \register{r}))\);
	\item \opcode{mul}/\opcode{mla} e simili sono per varie forme di moltiplicazione \footnote{Non esistono comandi per la divisione in assembly ARM};
	\item \opcode{b}/\opcode{bl} rispettivamente \emph{branch} e \emph{branch and link} (da notare che è possibile aggiungere qualsiasi suffisso condizionale);
	\item \opcode{bx} è implementato solo in alcune CPU, corrisponde a \mintinline{asm}{move r15, r};
	\item \opcode{cmp} setta i flags comparando due operandi, ma senza risultato, da usarsi prima di salti condizionali. Funzionano similmente anche \opcode{tst} (attraverso un and) e \opcode{teq} (attraverso uno xor);
	\item \opcode{ldr}/\opcode{str} sono rispettivamente \emph{load register} e \emph{store register}. Rispecchiano rispettivamente \opcode{lw} e \opcode{sw} dell'assembly MIPS;
	\item \opcode{mov}/\opcode{mvn}: \emph{move} equivalente all'omonimo MIPS e \emph{move not}, che sposta il complemento a 1 del registro sorgente (utilizzata per implementare il not).
\end{itemize}
\subsection{Possibilli operazioni sull'operando <r*>}\label{sec:modifiche}
Ecco un esempio di un'operazione in varie versioni diverse, per capire meglio tutte le possibilità di modifica offerte da ARM:
\begin{minted}[linenos, tabsize=4]{asm}
add r0, r1, #1
adds r0, r1, r2
addeq r0, r1, r2, lsl #2
addeqs r0, r1, r2, lsl r3
\end{minted}
Analizziamo le righe di codice una ad una:
\begin{enumerate}
	\item \register{r0} \(=\) \registerOp{r1}{+}{1};
	\item \register{r0} \(=\) \registerOp{r1}{+}{r2} (e setta dei flag se necessario);
	\item \register{r0} \(=\) \registerOp{r1}{+}{r2} (esegue la add solo se il flag eq è settato, DOPO l'operazione esegue uno shift a sinistra di 2 del registro r2);
	\item \register{r0} \(=\) \registerOp{r1}{+}{r2} (esegue la add solo se il flag eq è settato, setta falg se necessario, DOPO l'operazione esegue uno shift a sinistra di r3 del registro r2).
\end{enumerate}

\subsection*{Istruzioni di load/store multiple}
L'Assembly \emph{ARM} permette il salvataggio in memoria o il caricamento da memoria di più registri con una sola istruzione, \mintinline{asm}{ldm} e \mintinline{asm}{stm} per l'appunto. Le istruzioni appena citate hanno la seguente struttura:
\begin{center}
	\texttt{ldm [-mode][-cond] rb[!], (lista dei registri)}
\end{center}
dove \register{rb} è il registro che contiene la base, \suffix{-cond} è l'eventuale suffisso condizionale di cui si è già discusso in precedenza, mentre il suffisso (facoltativo) \suffix{mode} può essere:
\begin{itemize}
	\item \suffix{ia}: \emph{increment after}, incrementa il registro base dopo l'esecuzione di \opcode{ldm};
	\item \suffix{ib}: \emph{increment before}, incrementa il registro base prima dell'esecuzione;
	\item \suffix{da}: \emph{decrement after}, decrementa il registro base dopo l'esecuzione;
	\item \suffix{db}: \emph{decrement before}, decrementa il registro base prima dell'esecuzione.
\end{itemize}

Il \suffix{!}, facoltativo anch'esso, serve per rendere definitive le modifiche operate dai suffissi appena elencati, dopo l'operazione.

\section{Modalità di indirizzamento in ARM}
In modo semplificato esistono due metodologie di indirizzamento in ARM, ovvero \(\textrm{base} + \textrm{offset}\) (immediato) o \(\textrm{base} + \textrm{indice}\) (eventualmente scalato). La struttura dell'istruzione è la seguente:

\begin{minted}[tabsize=4, linenos]{asm}
ldr rd, rb (offset)           # base + offset
ldr rd, rb (indice) [shift]   # base + indice shiftato
\end{minted}
dove \register{rd} è il registro destinazione, \register{rb} è il registro contenente l'indirizzo base, \register{offset} è un immediato codificato su 12 bit e \register{indice} è un valore in registro, eventualmente shiftato o ruotato.

Una possibilità molto interessante offerta da \emph{ARM} è l'utilizzo dei meccanismi di \emph{pre} e \emph{post indexing}, vediamo come funzionano:

\begin{itemize}
	\item pre-indexing con offset: \texttt{[rb, \#i]!} in questo caso la somma tra \register{rb} e \register{i} viene  effettuata prima dell'indirizzamento, il \register{!} è facoltativo, se è presente finalizza la somma (il valore di \register{rb} dopo l'esecuzione è dato da \registerOp{rb}{+}{i});
	\item post-indexing con offset: \texttt{[rb], \#i!} in questo caso la somma tra \register{rb} e \register{i} viene effettuata dopo l'indirizzamento, il \suffix{!} se è presente finalizza la somma (il valore di \register{rb} dopo l'esecuzione è dato da \registerOp{rb}{+}{i});
	\item pre-indexing con indice e shift: \texttt{[rb, ri, shift]!} in questo caso la somma tra \register{rb} e \register{ri} shiftato (si ricorda che lo shift è facoltativo) viene  effettuata prima dell'indirizzamento, il \suffix{!} se è presente finalizza la somma;
	\item post-indexing con indice e shift: \texttt{[rb], ri, shift !} in questo caso la somma tra \register{rb} e \register{ri} eventualmente shiftato viene effettuata dopo dell'indirizzamento, \emph{"il \suffix{!} non fa molto, ma è un lavoro onesto"} -semicit.
\end{itemize}

Tirando le somme sui metodi di indirizzamento, si può dire che sono senz'altro più potenti di quelli offerti da \emph{MIPS}, soprattutto grazie alla possibilità di aggiornare automaticamente il registro di base, che diventa super comoda quando si devono gestire degli array. Rispetto ad \emph{Intel} invece \emph{ARM} ha la limitazione di non poter usare contemporaneamente offset ed indice.

\end{document}
