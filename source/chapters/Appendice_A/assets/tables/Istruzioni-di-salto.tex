\documentclass{standalone}

\usepackage{booktabs}
\usepackage{tabularx}
\usepackage[table]{xcolor}

\setlength\lightrulewidth{0.1pt}
\newcommand\lightrule{%
	\arrayrulecolor{black!30}%
	\midrule[\lightrulewidth]%
	\arrayrulecolor{black}}

\newcommand\register[1]{%
	\texttt{#1}%
}

% arara: pdflatex
% arara: latexmk: { clean: partial }
\begin{document}
\begin{tabularx}{\textwidth}{ >{\hsize=.11\textwidth}X >{\hsize=.18\textwidth}X >{\hsize=.08\textwidth}X X }
	\toprule
		Istruzione & Sintassi & Tipo & Funzione \\
	\midrule
		j & j C & J & jump: salta a \(C\cdot4\) \\\lightrule
		jal & jal C & J & jump and link: salta a \(C\cdot4\) e salva il valore di PC in \register{\$ra} \\\lightrule
		jr & jr \register{\$r} & R & jump register: salta all'indirizzo contenuto in \register{\$r} \\\lightrule
		beq & beq \register{\$s}, \register{\$t}, C & I & branch equal: se \register{s} e \register{t} sono uguali, salta a  \(PC + 4 + C \cdot 4\) \\\lightrule
		bne & bne \register{\$s}, \register{\$t}, C & I & branch not equal: se \register{s} e \register{t} sono diversi, salta a  \(PC + 4 + C \cdot 4\) \\
	\bottomrule
\end{tabularx}
\end{document}
