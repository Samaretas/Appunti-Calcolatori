\documentclass[class=book, crop=false]{standalone}
\usepackage[subpreambles=true]{standalone}

\usepackage{../../style}

\graphicspath{{./assets/images/}}

\begin{document}
\chapter{L'architettura MIPS}

\section{Istruzioni aritmetico-logiche}

\subsection{Operazioni di shift}

\paragraph{Shift logico a sinistra}
Consideriamo lo \emph{shift logico a sinistra}: l'idea consiste nell'inserire degli zeri nella posizione meno significativa e traslare tutto a sinistra perdendo, nel caso di overflow, i bit più significativi.\\
\emph{ssl \$t2, \$s0, 4}: Memorizza in t2 il contenuto del registro s0 shiftato a sinistra di 4.\\
Lo shift massimo che è possibile eseguire è di 32 bit (infatti vengono allocati 5 bit per il campo di \emph{shift amount}). Ciò non risulta un problema in quanto le istruzioni \emph{MIPS} che affronteremo sono codificare in 32 bit.\\
Eseguire uno shift a sinistra di \(k\) bit di un numero \(n\) positivo, consiste nel moltiplicare \(n\) per \(2^k\). Fanno eccezione solo alcuni casi in cui si lavora con numeri interi codificati in complemento a 2: in questo caso si esce dal \emph{range di rappresentazione} del CA2.

\paragraph{Shift logico a destra}
In maniera analoga, lo \emph{shift logico a destra} consiste nell'inserire degli zeri nella posizione più significativa, traslando tutto a destra. Come per lo shift logico a sinistra, anche qui è possibile effettuare uno shift di al massimo 32 bit.\\
Lo \emph{shift logico a destra} di \(k\) bit significa implementare la divisione intera per \(2^k\) per numeri positivi. In questo caso, non è utilizzabile con nessun numero negativo codificato in CA2 (proprio per questo è stato introdotto lo shift aritmetico a destra).\\
\emph{srl \$t2, \$s0, 1}

\paragraph{Shift aritmetico a destra}
Invece che inserire 0 come bit più significativo, si ricopia il bit di segno. Dunque risolve i problemi con CA2.

% TODO - tabella riassunta operazioni logiche (shitf sx, shift dx, bitwise and, bitwise or, bitwise not)

\subsection{Operazione di bitwise}
\paragraph{Bitwise AND}
L'AND è utilizzato come maschera.\\
\emph{and \$t0, \$t1, \$t2}

\paragraph{Bitwise OR}
\emph{or \$t0, \$t1, \$t2}

\paragraph{Bitwise NOR}
Operazione di NOT OR bit a bit\\
\emph{nor \$t0, \$t1, \$t2}\\
Il not non è implementato in mips, altrimenti romperebbe la regolarità delle istruzioni a tre operandi. Un modo per ottenere il not, grazie alle leggi di De Morgan:\\
\emph{nor \$t0, \$t1, \$zer}

\paragraph{OR esclusivo XOR}
Produce 1 se i due bit di ingresso sono uguali, altrimenti zero se sono uguali. Faro uno xor di un numero per se stesso: annulla il registro, resettandolo tutto a 0.
\emph{xor \$t0, \$t1, \$t2}

\section{Salti}
In inglese c'è differenza fra \emph{jump} (non condizionato) e \emph{branch} (condizionato). Il branch modifica il valore del registro \emph{PC}.\\
\emph{beq reg1, reg2, L1}: Branch on EQual.\\
\emph{bne reg1, reg2, L1}: Branch on Non Equal.\\
\emph{j LABEL}: salta all'indirizzo di label\\
\emph{jr \$t0}: salta all'indirizzo contenuto nel registro
Le label vengono alla fine tradotte in indirizzi da parte del compilatore.
Il check del branch viene fatto con la condizione opposta, per aver maggiore efficienza.

\subsection{Altri confronti}
\emph{slt \$t0, \$s3, \$s4}: t0=1 se s3<s4.\\
\emph{slti \$t0, \$s3, 10}: t0=1 se s3<10.\\

% TODO - Riguarda la parte relativa al segno

\subsection{Implementazione del costrutto \emph{if}}

\subsection{Implementazione del costrutto \emph{while}}
Il ciclo non viene direttamente implementato dal mips.\\
% TODO - Codice C - Codice MIPS (magari affiancati con minipage)

\subsection{Implementazione del costrutto \emph{switch/case}}
\begin{itemize}
	\item Possibilità di creare una sequenza di if-else
	\item
\end{itemize}

% \subimport{./assets/listings/}{test.tex}
\begin{minted}[linenos,breaklines]{asm}
# somma di due registri
add $t0, $s1, $s2;
add $t1, $s3, $s4;

# lettura e scrittura in memoria
lw $t0, 4($s3);
sw $to, 48($s3);

# somma di una costante ad un registro
\end{minted}
\begin{minted}[linenos,breaklines]{asm}
# somma di due registri
add $t0, $s1, $s2;
add $t1, $s3, $s4;

# lettura e scrittura in memoria
lw $t0, 4($s3);
sw $to, 48($s3);

# somma di una costante ad un registro
\end{minted}

\end{document}
