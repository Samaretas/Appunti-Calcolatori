\documentclass[class=book, crop=false, oneside]{standalone}
\usepackage[subpreambles=true]{standalone}

\usepackage{../../style}

\graphicspath{{./images/}}

% arara: pdflatex: { synctex: yes, shell: yes }
% arara: latexmk: { clean: partial }
%! arara: clean: { extensions: [sta] }
\begin{document}
\chapter*{Introduzione}
% \addcontentsline{toc}{chapter}{Introduzione} \markboth{INTRODUZIONE}{}

Quella che il lettore sta approciando è una dispensa che raccoglie, riorganizza e riespone l'intero contenuto del corso di \emph{calcolatori} tenuto nell'Università degli studi di Trento durante il secondo semestre dell'anno accademico 2018/2019. La repository del progetto può essere raggiunta tramite il seguente link: \url{https://github.com/Samaretas/Appunti-Calcolatori}.

Le risorse utilizzate sono state le slides fornite dal professor Giovanni Iacca (corso dispari) durante le lezioni e integrate dalle sue spiegazioni, la dispensa "Piccola introduzione al linguaggio Assembly e simili amenità" del professor Luca Abeni, le lezioni dell'esercitatore Fabiano Zenatti e svariate risorse online per dirimere alcune lievi questioni non meritevoli del disturbo ai professori. L'elaborato è stato suddiviso in capitoli che seguono il partizionamento proposto dal professore, e successivamente in sezioni e sottosezioni per questioni di agilità consultativa. I singoli argomenti vengono presentati nell'ordine e modalità del professore (all'infuori di sporadici casi in cui gli autori hanno proposto una maniera da loro ritenuta più fruibile per il target di riferimento), corredati da immagini e codici commentati dove possibile.\\

Nel momento della stesura (ma gli autori hanno ragione di affermare che varrà anche per il futuro prossimo), questa dispensa si propone di essere autoconsistente nell'ottica della preparazione dell'esame di fine corso, ossia lo studente potrebbe potenzialmente usare questo testo come unica risorsa ed essere ampiamente in grado di superare con successo l'esame. Quest'ultimo, alla data attuale, consiste in 12 domande a risposta multipla, ciascuna delle quali del valore di \(2.75\) punti, per un totale di \(33\). Si tenga presente che ogni risposta errata comporta una decurtazione di \(0.55\) punti dal totale; risposte considerate nulle non comporteranno alcuna penalità.\\

Ecco una breve presentazione degli autori; per raggiungere il profilo GitHub di ciascuno è sufficiente cliccare sul relativo nome; ognuno dei tre, a turno, ha curato l'esposizione dei singoli capitoli.
\begin{itemize}
  \item \emph{\href{https://github.com/FrancescoBozzo}{Francesco Bozzo}}: leader e responsabile del progetto, ha frequentato il liceo scientifico \sout{da qualche parte a Vicenza dove abita e mangia gatti} e successivamente si è iscritto all'Università di Trento, corso di laurea in informatica. Nel progetto si è dedicato principalmente alla sua strutturazione a livello di codice, andando a ricercare di volta in volta il pacchetto più appropriato per presentare il capitolo esattamente nella maniera migliore, e risolvendo ogni volta qualsiasi problema \LaTeX\ potesse presentare, il che gli ha giustamente procurato il titolo di campione della Lega Pokémon della regione di Trentoh.
  \item \emph{\href{https://github.com/Samaretas}{Samuele Conti}}: ha frequentato scienze applicate a Verona, quindi non un liceo vero, e da un anno a questa parte frequenta l'università di Trento dove ha raggiunto la notorietà per mezze delle sue memorabili sbronze. Particolarmente notevoli i suoi errori di ortografia e la sua incapacità di formulare pensieri in modo ordinato. Tutto sommato però è un bravo ragazzo, ve lo raccomando.
  \item \emph{\href{https://github.com/filippodaniotti}{Filippo Daniotti}}: dopo aver frequentato il liceo classico Antonio Scarpa, si è iscritto al corso di laurea in informatica presso l'Università di Trento, già da questa premessa si deduce quindi il suo forte carattere masochista. Ha curato la revisione ortografica e morfosintattica del progetto, il commento dei codici e la stesura di quest'introduzione. Inoltre ha sempre dimostrato una grande passione per l'utilizzo del tool \LaTeX\ (qui ritorna il carattere masochista) e molti ritengono che per questa ragione sia il classico tipo che alle feste rimane solo in un angolino a compiangere se stesso.
\end{itemize}

\end{document}
