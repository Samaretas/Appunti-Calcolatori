\documentclass[class=book, crop=false, oneside]{standalone}
\usepackage[subpreambles=true]{standalone}

\usepackage{../../style}

% arara: pdflatex: { synctex: yes, shell: yes }
% arara: latexmk: { clean: partial }
%! arara: clean: { extensions: [sta] }
\begin{document}
\chapter{Come compilare Assembly Intel}

\section{Prerequisiti}
Il sistema operativo che utilizzeremo è una qualsiasi distribuzione Linux. A seguire i software che andremo a utilizzare:
\begin{itemize}
	\item \emph{\acrfull{gcc}}: il compilatore C GNU;
	\item \emph{ld}: il linker GNU;
	\item \emph{nasm}: un assemblatore Intel x86.
\end{itemize}

\subsection{Come utilizzare \acrshort{gcc}}
Per compilare il file C "main.c" attraverso \acrshort{gcc} nell'eseguibile "main":
\begin{minted}{batch}
	gcc main.c -o main
\end{minted}
mentre per compilare un file C in assembly AT\&T attraverso \acrshort{gcc}:
\begin{minted}{batch}
	gcc -S main.c -o main.s
\end{minted}

\subsection{Come utilizzare nasm e ld}
Per compilare un file Assembly Intel "pluto.asm" nel file oggetto "pluto.o":
\begin{minted}{batch}
	nasm -felf64 -o pluto.o pluto.asm
\end{minted}
e successivamente per linkare il file oggetto e ottenere l'eseguibile "pluto":
\begin{minted}{batch}
	ld -m elf_i386 -o pluto pluto.o
\end{minted}

\section{Un po' di esempi}
Esistono principalmente due sezioni: \mintinline{gas}{.data} che racchiude eventuali dati da salvare in memoria e \mintinline{gas}{.text} che contiene il codice Assembly. In particolare, in quest'ultima sezione bisogna specificare un'etichetta come \mintinline{gas}{global}: l'etichetta che andiamo a specificare indicherà l'inizio del programma.

\subsection{Hello World}
\begin{minted}[linenos, breaklines, tabsize = 4]{gas}
SECTION .data
msg db "Hello World", 0Ah  # 0A permette di andare a capo riga

SECTION .text
global _start

_start:
	mov edx, 13   # lunghezza stringa
	mov ecx, msg  # puntatore alla stringa
	mov ebx, 1    # seleziona come output lo std output
	mov eax, 4    # system call per stampare
	int 80h

	mov ebx, 0    # return 0
	mov eax, 1    # sys call close
	int 80h
\end{minted}
Da notare che se non si inserisce l’ultima sys call (\mintinline{batch}{return 0}) verrà prodotto un errore di segmentation fault.

\subsection{Stampa di una stringa di lunghezza variabile}
\begin{minted}[linenos, breaklines, tabsize = 4]{gas}
SECTION .data
msg db "Hello World", 0Ah, 0h  # 0h terminatore di stringa

SECTION .text
global _start

_start:
	# esegue un ciclo while
	mov ebx, msg
	mov eax, ebx
nextchar:
	cmp byte[eax], 0
	jz finished
	inc eax
	jmp nextchar
finished:
	sub eax, ebx
	# stessa cosa di prima, semplicemente la lunghezza è variabile in eax
	mov edx, eax
	mov ecx, msg
	mov ebx, 1
	mov eax, 4
	int 80h

	mov ebx, 0
	mov eax, 1
	int 80h
\end{minted}

\subsection{Esempio di subroutine}
\begin{minted}[linenos, breaklines, tabsize = 4]{gas}
SECTION .data
msg db "Scritta", 0Ah, 0h

SECTION .text
global _start

_start:
	mov eax, msg
	call strlen

	mov edx, eax
	mov ecx, msg
	mov ebx, 1
	mov eax, 4
	int 80h
	mov ebx, 0
	mov eax, 1
	int 80h
strlen:
	push ebx
	mov ebx, ea
nextchar:
	cmp byte[eax], 0
	jz finished
	inc eax
	jmp nextchar
finished:
	sub eax, ebx
	pop ebx
	ret
\end{minted}

\end{document}
