\chapter{Codifica del testo}
\section{La codifica ASCII}
Oltre alle codifiche per numeri naturali, interi e reali viste nel capitolo precedente, sono state introdotte delle codifiche per rappresentare sequenze di caratteri. Una delle più note è la codifica \textbf{ASCII} (\textit{American Standard Code for Information Interchange}.), che utilizza \(7\) bit per rappresentare ciascun carattere (quindi \(2^7=128\) caratteri disponibili in totale).\\
Attraverso la codifica \textit{ASCII} è possibile rappresentare tutte le lettere dell'alfabeto anglosassone (lettere maiuscole, minuscole, numeri, punteggiatura, ecc.).\\
Questa codifica presenta uno svantaggio: seppur utilizzi solamente 7 bit, un byte è formato da 8 bit. Ciò implica che il primo bit di ciascun carattere viene impostato a 0 per default.

\section{La codifica ASCII estesa}
A differenza dell'\textit{ASCII} tradizionale, l'\textit{extended ASCII} sfrutta anche l'ottavo bit per codificare caratteri addizionali (quindi \(2^7=128\) caratteri disponibili in totale).\\
Non esiste un unico standard esteso: sono state implementate diverse versioni, ciascuna in grado di supportare diversi alfabeti specifici. In particolare, i byte con valore minore di \(128\) sono comuni a tutte le implementazioni della codifica ASCII estesa.

\section{La codifica UNICODE}
La necessità di avere una codifica univoca ha portato ad aumentare fino a \(32\) il numero di bit per la rappresentazione di un carattere attraverso la codifica \textbf{Unicode}. Anche qui non mancano differenti formati di codifica:
\begin{itemize}[noitemsep]
	\item \textit{UTF-32}: ogni simbolo è composto da 32 bit.
	\item \textit{UTF-16}: ogni simbolo è composto da 16 o più bit (lunghezza variabile).
	\item \textit{UTF-8}: ogni simbolo è composto da 8 o più bit; mantiene la compatibilità con ASCII.
\end{itemize}

\paragraph*{Input e output da file}
Come si è visto precedentemente, è dunque possibile codificare dati numerici in diverse modalità. Quando si scrive un programma che si interfaccia con file testuali, si consiglia caldamente di salvare i dati numerici in formato binario e non in forma di stringa. Questa accortezza permette di ridurre la dimensione in byte del file finale prodotto.

